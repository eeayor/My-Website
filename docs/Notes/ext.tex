\documentclass{book}
\usepackage{times}
\author{Logan Jackson}
\title{Existentialism}
\date{December 19th, 2019}
\begin{document}
\maketitle
\part{Introduction}
\section{Topics in Existential Thinking}
\paragraph{Nihilism}
So let's begin with the most common and popular conclusion from what people think is existentialism. Although Nihilism isn't really existentialism it's not really the same things though the Nihilists do takes some of the postulates of Existentialism. The difference here is the way each ideology interprets them. So for example: someone who is in the camp of nihilism believes that we are floating around the world without purpose this is fundamentally a bad thing due to the fact that, since essence precedes existence, we have no essence. Therefore, nothing really matters and people just live such that they can die. Existentialists believe that, since essence precedes existence, we as humans are the ones who create meaning in the first place since we are not given any essence it's our job as the logical and rational beings we are to form the world of essence around us. Since nature and the universe is fundamentally random; we cannot control it but we can control our place in it through our own decisions and actions which we are completely responsible for. Free will is a fundamental postulate of this idea. If humans do not have free will then Existentialism is then invalid. If we do have free will then Existentialism is valid.
\paragraph{Freedom}
Freedom and existentialism, the idea behind freedom and existentialism is that we are fundamentally free beings therefore, most of this pain and terror that we feel is fundamentally over the issue of freedom. We are burdened by choice itself and can make one of two decisions in reaction to this realization we can either a. Make no choice or b. Choose to choose and live the life that we want to. 
\paragraph{Ontology}
Ontology is the study of being. Existentialism deals with this frequently and thoroughly considering that Existentialism is itself somewhat of a phenomenological study of being. Ontological study of being. When we look at ontology as it relates to Existentialism we see that Existentialist philosophy brings some new interpretations to the table when it comes to ontology. Most notably on the ideas of choice and free will. If we are all burdened by choice then why do most of us choose not to choose. Conformity is an issue in our society and many people are on that conforming type beat and it's really somewhat interesting to observe the behavior of said people. When we look at ontology we must look at it in a post modern context, due to the nature of our current understanding of society and the philosophy of groups. When we look at Existential ontology in this fashion we get our first conflict which is the idea that we are not possibly free or we cannot possibly have free will due to the very conception of society.

\section{Existential Thinkers}
\paragraph{Jean Paul Sartre}
Perhaps the most well known and most influential of all of the Existentialist thinkers on the topic of existentialism. Sartre lays the frame work for which we now think of as existentialism.
\end{document}
